\documentclass[12pt]{letter} % Uses 10pt

\usepackage[portuguese]{babel}
\usepackage[utf8]{inputenc}
\usepackage{merriweather}
\usepackage{multirow}
\usepackage{graphicx}
\usepackage{microtype}

\topmargin=-1in    % Make letterhead start about 1 inch from top of page 
\textheight=8in  % text height can be bigger for a longer letter
\oddsidemargin=0pt % leftmargin is 1 inch
\textwidth=6.5in   % textwidth of 6.5in leaves 1 inch for right margin

\begin{document}

\signature{\includegraphics[width=5cm]{assinatura_blue}\\
Alejandro C.\ Frery\\
Professor Titular}           % name for signature 
\longindentation=0pt                       % needed to get closing flush left
\let\raggedleft\raggedright                % needed to get date flush left
 
\begin{letter}{Prêmio Beatriz Neves\\
SBMAC}
\date{14 de abril de 2019}

\begin{flushleft}
Alejandro C.\ Frery
\end{flushleft}
\smallskip\hrule height 2pt
\begin{flushright}
\begin{tabular}{rl}
\multirow{3}{*}[-1.3em]{\includegraphics[width=3.5cm]{laccan.pdf}}	\\
& \small LaCCAN -- Laboratório de Computação Científica e Análise Numérica\\
	& \small CPMAT -- Centro de Pesquisa em Matemática Computacional\\
	& \small Universidade Federal de Alagoas\\
	& \small Av. Lourival Melo Mota, s/n, Cidade Universitária\\
	& \small  57072-900 Maceió, AL -- Brazil
\end{tabular}
\end{flushright} 
\vfill % forces letterhead to top of page

\opening{} 
Esta correspondência tem por objetivo encaminhar o trabalho de Iniciação Científica de Eduarda Tatiane Caetano Chagas intitulado ``Teoria da Informação e Estatística Computacional
no Processamento e Análise de Sinais -- Uma
ferramenta para Análise de Séries Temporais''.
Esta monografia foi desenvolvida sob a minha orientação, como parte dos requisitos para conclusão do curso em Ciência da Computação pela Universidade Federal de Alagoas.

O ponto de partida desse trabalho foi o desenvolvimento de ferramentas ami\-gá\-veis e numericamente confiáveis para a análise de séries temporais empregando Teoria da Informação.
Eduarda precisou estudar artigos científicos de conteúdo e profundidade que vão muito além do que se espera de um aluno de graduação.
Ela não fez apenas isso, mas assimilou os conceitos e foi capaz de desenvolver a plataforma computacional incorporando resultados (como os de imputação de dados repetidos) que acabaram de aparecer na literatura.

A contribuição de Eduarda se tornou fundamental para facilitar as pesquisas nesta área.
Convém frisar que este trabalho é interdisciplinar, pois envolve Teoria da Informação, Estatística Computacional, Análise Numérica, Processamento de Sinais e Desenvolvimento de Software Científico.

Ela concluiu a sua graduação com uma formação diferenciada e uma excelente experiência de trabalho de pesquisa.

Pelo seu desempenho como aluna e orientanda,
pela qualidade do seu trabalho, e
pela relevância do produto que ela desenvolveu,
recomendo que ela seja considerada para receber o Prêmio Beatriz Neves da Sociedade Brasileira de Matemática Aplicada e Computacional.

\closing{Muito atenciosamente,} 
  
%\encl{Dois formulários 0008/SPT-UFAL;V.2}  				% Enclosures

\end{letter}

\end{document}






