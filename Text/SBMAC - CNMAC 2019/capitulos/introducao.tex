\mychapter{Introdução}{cap:introducao} \lhead{INTRODUÇÃO}

\section{Motivação}

Séries temporais estão presentes em todo o nosso cotidiano. São definidos como um conjunto de dados obtidos a partir de um processo observacional ao longo de um determinado período de tempo, não necessariamente dividido em espaços iguais, caracterizados pela dependência serial existente entre as observações.

A hipótese subjacente a toda essa análise é que os dados observados são o resultado da operação de um sistema causal sujeito a ruído observacional. Logo, esse sistema, ou dinâmica, é responsável pela criação de padrões que através de observações podemos inferir a respeito da dinâmica. Portanto, o estudo de tais dados auxilia na análise de diversas propriedades de sistemas. 

Como comentado anteriormente, a aplicação deste conhecimento pode ser encontrada em múltiplas áreas do conhecimento científico como, por exemplo, 
na discriminação entre fenômenos estocásticos e caóticos~\citep{DistinguishingNoiseFromChaos}, 
na identificação de padrões de comportamento em redes veiculares~\citep{CharacterizationVehicleBehaviorInformationTheory}, 
na classificação e verificação de assinaturas \textit{online} ~\citep{ClassificationVerificationOnlineHandwrittenSignatures},
na análise da eficiência informacional do mercado de petróleo~\citep{oilMarket},
na caracterização das séries temporais produzidas por eletroencefalogramas~\citep{EGGTimeSeries},
na análise da robustez de redes~\citep{InformationTheoryPerspectiveNetworkRobustness}, e 
na classificação de padrões de consumo de energia elétrica~\citep{CharacterizationElectricLoadInformationTheoryQuantifiers}.

Tradicionalmente o estudo de séries temporais costuma ser dividido em duas linhas de estudo, nos domínios do tempo e da frequência~\citep{BrockwellDavis91}. No entanto, ambas abordagens utilizam diretamente os dados resultantes do processo observacional, que são sensíveis a efeitos provocados por diversos tipos de contaminação. Logo, surge assim a abordagem do uso de métodos não-paramétricos, como uma forma de evitar que tais efeitos invalidem as análises destes dados.

A Teoria da Informação surgiu como um ramo interdisciplinar, produzindo inúmeros resultados, tanto no ponto de vista teórico quanto nas aplicações e criação de novos métodos de extração de informações em sinais, abrangendo em suas soluções conceitos presentes na Probabilidade, Estatística e Telecomunicações. 

O uso de suas ferramentas tem levado a resultados significativamente melhores do que aqueles obtidos com técnicas clássicas em diversas áreas do conhecimento. No trabalho de~\cite{Torres2014}, podemos ver uma grande contribuição no campo de processamento de imagens, onde este propõe uma técnica de filtragem que se adapta a cada ponto da imagem, observa uma janela de tamanho considerável e só emprega aquelas observações que não são muito discrepantes do valor central. Em~\cite{Bhattacharya2015}, vemos uma aplicação de distâncias estocásticas para obter uma decomposição polarimétrica otimizada. Já~ \cite{Gambini2015} propõe uma técnica de estimação de parâmetros minimizando distâncias estocásticas entre modelos e evidência empírica.

Entretanto, diversos desafios surgem na hora de tratar um problema com estes tipos de técnicas, pois ainda existem vários problemas analíticos e de ordem computacional em aberto, formando assim uma linha de pesquisa avançada, uma vez que requerem por parte dos envolvidos um bom domínio das teorias que dão sustentação às técnicas.

Atualmente há diversas ferramentas que auxiliam na análise clássica de séries temporais; para a plataforma \texttt  R, existem diversas bibliotecas para essa finalidade (ver \url{https://cran.rproject.org/web/views/TimeSeries.html}). Além destas opções, o usuário também pode contar com os softwares de visualização de séries temporais. No entanto, todas alternativas são limitadas as opções de bibliotecas e softwares que trabalham, em sua grande parte, com técnicas paramétricas e exigem familiaridade do usuário com o ambiente utilizado.

Desse modo, exitem dois principais pontos nessas linhas de pesquisa que podem originar ótimos trabalhos inovadores:

\begin{itemize}
\item a necessidade de tornar as técnicas acessíveis a usuários não especializados, e
\item a necessidade de otimizar o desenvolvimento de novas técnicas.
\end{itemize}

O primeiro ponto pode ser solucionado por meio do desenvolvimento de sistemas com interface gráfica que encapsulem os algoritmos presentes na literatura. Já o segundo, consiste em utilizar técnicas de desenvolvimento de software científico.

Logo, é na esfera do domínio dos problemas computacionais que surgem na aplicação de ferramentas oriundas da Teoria de Informação a séries temporais, que este trabalho se insere.
 
Apresentamos, assim, o desenvolvimento de uma ferramenta portável, rápida e de boa qualidade numérica que possibilita análises interativas e exploratórias dos dados de uma série temporal através de técnicas provenientes da Teoria da Informação.
Com ela, o usuário dispõe de um conjunto de técnicas de análise presentes na literatura para processar e examinar seus dados de modo eficiente e com um mínimo período de aprendizado.
A ferramenta é extensível.

\section{Objetivo}

O objetivo geral deste trabalho é propor e desenvolver uma ferramenta inovadora, resultante de propostas recentes de pesquisas relacionadas a Teoria da Informação, para facilitar o uso de técnicas avançadas de processamento e análise de sinais.

\section{Solução proposta}

Realizamos o uso de técnicas modernas de análise de séries temporais. 
Uma série temporal é transformada em uma sequência de símbolos, através da técnica de simbolização de~\cite{article2}. 
Essa técnica consiste em transformar vetores de tamanho $D$ em padrões ordinais de forma não-paramétrica e formar um histograma de ocorrência dos $D!$ padrões possíveis. 
Esse histograma é tratado como uma função de probabilidade, do qual são extraídos descritores oriundos da Teoria da Informação. Esses descritores são, depois, mapeados em um plano adequado, e a sua localização serve para identificar o tipo de dinâmica subjacente à série temporal. 
Há uma grande diversidade de descritores como, por exemplo, distâncias (Kullback-Leibler, Bhattacharya, Hellinger, Rényi, Triangular, Harmônica, dentre outras), e entropias (Jensen-Shannon, Rényi, Tsallis, dentre outras). 
O ambiente gráfico oferecerá essas opções, e permitirá experimentar com a sua expressividade.

\section{Contribuições}

As contribuições deste trabalho são:

\begin{itemize}
\item A compreensão e implementação de técnicas de análise não-paramétrica de séries temporais utilizando descritores causais oriundos da Teoria da Informação;
\item A implementação de uma interface gráfica amigável para a aplicação de tais descritores, mantendo a portabilidade do software para os diversos sistemas operacionais e arquiteturas de hardware.
\end{itemize}

Note que essas contribuições podem facilitar este processo de análise e construção do conhecimento por parte do usuário, tornando tal experiência mais simples e completa, fornecendo para este novas funcionalidades e uma maior interação do gráfico da série com os seus padrões.

\section{Estrutura do texto}

Este trabalho foi dividido em 5 capítulos e um anexo. 
No capítulo~\ref{cap:fundamentacao} introduzimos algumas principais técnicas e ferramentas disponíveis na literatura para a análise não-paramétrica de séries temporais utilizando descritores da Teoria da Informação, focando nos conceitos e metodologias aplicados com sucesso em diversos ramos de pesquisa científica.
No capítulo~\ref{cap:metodologia} apresentamos a metodologia do trabalho desenvolvido.
No capítulo~\ref{cap:resultados} mostramos os resultados obtidos.
As funções implementadas ao longo do desenvolvimento do projeto se encontram presente no Anexo A.
E, finalmente, no Capítulo~\ref{cap:conclusoes} apresentamos as considerações finais, concluindo este trabalho.

\newpage\lhead{\rightmark}
